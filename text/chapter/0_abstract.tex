\section*{Abstract}
%General Background:
Image generation models, such as Stable Diffusion, are popular tools for creating any desired image from a line of text. 
%Specific Background:
The inclusion of new transformer-based technologies into these models has improved the quality of their images, though their operating time can be very slow.
%Knowledge Gap:
\cite{bolya2023tomesd} have introduced Token Merging (ToMe) for Stable Diffusion to improve computational time by reducing the number of tokens evaluated by the model.
%Here we show...:
The goal of this thesis was to replicate their results and investigate whether performance can be improved by using different configurations for ToMe.
%Results:
Our results show that ToMe's default configuration maintains great image quality (most of the time), while accelerating image generation by up to $2.2 \times$, when reducing the number of tokens by up to \(60\%\). Beyond that, we identified different configurations for ToMe that improve the performance compared to the default setup, by using different partitioning methods and expanding the scope of token merging in the transformer. 
%Implications:
This work further solidifies the viability of token merging in diffusion models and incentivises further research into improving the efficiency of image generation algorithms.
Code and full-scale images are available at \url{https://github.com/HNR1/ba-code}
